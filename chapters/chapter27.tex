\chapter{宇宙的结构}
\section{系外距离测量}
如果距离的尺度大于银河系的尺度时,这时就被称为系外距离尺度或\textbf{宇宙学距离尺度}(Mpc量级)。对于这种尺度的距离测量,三角视差的误差会比较大,因此需要引入其他的测距方法,其中最著名的称为\textbf{标准烛光}。标准烛光指的是一类我们已知光度的天体,而通过测量它们的视星等,我们就可以得到它们的距离。

\paragraph{造父变星}
由于造父变星具有周光关系,当我们测量到它的周期,也就意味着知道了它的光度
\begin{equation}
  M_{\langle V\rangle}=-3.53\log_{10}P_d-2.13+2.13(B-V)
\end{equation}

目前观测到的最远造父变星可以达29\,Mpc。

\paragraph{Ia型超新星}
前面提到Ia型超新星的产生是由于白矮星吸积物质达到了钱德拉塞卡极限,因此爆发时具有一个统一的光度。\mbox{}\\

其他的能够用于距离探测的\textbf{示距天体}还包括:新星(20\,Mpc)、最亮的红巨星(7\,Mpc)、球状星团、行星状星云和塔利-费希尔关系等。

\section{宇宙的膨胀}
\textbf{哈勃定律}告诉我们所有的星系都是在远离我们运动的,并且退行速度和距离成正比
\begin{equation}
  v=H_0d
\end{equation}

其中$H_0=100h\;\mathrm{km\,s^{-1}\,Mpc^{-1}}$是\textbf{哈勃常数},$h$的取值范围是0.5-1;现在通常取0.7,这显然也可以作为测距方式。哈勃定律还解释的一个现象则是宇宙的空间是在加速膨胀的,星系随着空间膨胀的运动被称之为\textbf{哈伯流},而引力束缚系统不随空间发生膨胀。

\paragraph{大爆炸理论}
如果宇宙是在膨胀,那么倒过来考虑,一开始宇宙应该只是一个奇点,之后一切的物质都起源于一场大爆炸,宇宙开始膨胀。而早期的宇宙温度非常高,充满了黑体辐射,随着宇宙的膨胀,温度逐渐降低,辐射峰值也会随之红移,最终变成了现在的微波波段的\textbf{宇宙微波背景}(cosmic microwave background),对应温度3\;K。