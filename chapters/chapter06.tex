\chapter{望远镜}
\section{基础光学}
\paragraph{斯涅尔定律}
表示折射的规律,$n_{1\lambda}\sin\theta_1=n_{2\lambda}\sin\theta_2$

\paragraph{透镜公式}
${1\over f_\lambda}=(n_\lambda-1)\left({1\over R_1}-{1\over R_2} \right)$,其中$R_1$和$R_2$是折射面的曲率,凸面为正,凹面为负。

透镜的\textbf{放大本领}可以用底片放大率(plate scale)表示
\begin{equation}
  {d\theta\over dy}={1\over f}
\end{equation}

可得透镜的焦距越大,相同角间距的两点能够呈现更大的像。但事实上无法通过简单的提高焦距来提升望远镜的性能,因为极小的角间距会产生衍射,因此存在一个受波长$\lambda$和光圈大小$D$(或孔距)限制的最小分辨角,表征透镜的\textbf{分辨本领}(瑞利判据):
\begin{equation}
  \theta_\mathrm{min}=1.22{\lambda \over D}
\end{equation}

\paragraph{视宁度}
望远镜显示图像的清晰度,取决于大气的湍流活动引起光线折射的程度。

\paragraph{像差}
透镜和镜面系统所带来的图像畸变称为像差(aberrations),包括
\begin{itemize}
  \item 色差,折射式望远镜由于折射率与波长有关,不同波长的光汇聚在不同焦平面上,可通过增加改正透镜来降低影响
  \item 球差,球面镜的边缘比中心具有更强折射能力,使得平行光无法汇聚在一点,可通过设计(抛物面)消除
  \item 彗差,抛物面镜对偏离光轴的光源的入射光不能汇聚在一点上,即焦距与入射光与光轴夹角有关,彗尾状的像而得名,可通过选择合适的表面曲率降低影响
  \item 散光,透镜或镜面成像不在预定焦平面上,属于设计缺陷
  \item 场曲,成像不在平面,而是在曲面上
  \item 场变,底片放大率与到光轴的距离相关
\end{itemize}

望远镜的\textbf{获取光线的本领}称为照度$J\propto 1/F^2$,其中\textbf{焦比}$F\equiv f/D$。

\section{望远镜}
\paragraph{折射式}——叶凯士天文台
\begin{itemize}
  \item 角放大率$m=f_\mathrm{obj}/f_\mathrm{eye}$
  \item 产生色差
  \item 边缘支撑时间久了会变形
  \item 热响应慢
  \item 容易有设计缺陷
\end{itemize}

\paragraph{反射式}——卡塞格林望远镜、施密特望远镜
\begin{itemize}
  \item 只需要注意维护反射面
  \item 主动光学
  \item 各种设计
\end{itemize}

\paragraph{主动光学(active optics)}
观测时温度变化和望远镜移动会导致镜面的形变,主动光学就可以通过在镜面后增加活塞改变镜面曲率,或每隔几分钟调整望远镜朝向来修正

\paragraph{自适应光学(adaptive optics)}
有了主动光学,观测误差来源主要就在大气湍流,自适应光学通过计算机可在曝光同时快速修正,通常可以在光路中插入一个小透镜,调整小透镜的曲率来调整焦点位置

\paragraph{望远镜支架}——赤道仪式、高度-方位式

\paragraph{射电望远镜}——FAST、VLA,
可通过增大面积(综合孔径)和干涉(VLBI)的方式提高分辨率。
\newline

由于大气窗口,红外、紫外、x射线和伽马射线的望远镜需要发射到太空。
\paragraph{红外望远镜}——Spitzer、Hershel

\paragraph{紫外望远镜}——IUE国际紫外望远镜、EUVE极紫外探测器

\paragraph{x射线望远镜}——Chandra、XMM-Newton

\paragraph{伽马射线望远镜}——Fermi、悟空