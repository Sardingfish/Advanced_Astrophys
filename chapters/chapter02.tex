\chapter{天体力学}
\section{牛顿力学}
牛顿三大定律+万有引力定律

联立动能和势能的关系$K=-U$,可得逃逸速度$v_{ves}=\sqrt{2GM/r}$

\section{开普勒定律}
在多体运动的研究中,使用{\bf 质心参考系}会更加方便,质心$\mathbf{R}\equiv {m_1 \mathbf{r}'_1+m_2
\mathbf{r}'_2 \over m_1+m_2} $可代替多体系统的整体运动情况,折合质量$\mu \equiv {m_1m_2 \over m_1
+m_2}$可表征系统的能量分布情况,因此两体的位移矢量$\mathbf{r_1}$和$\mathbf{r_2}$可表示为:
\begin{align}\label{eq:r}
  \mathbf{r_1} &=-{\mu\over m_1}\mathbf r\\[2mm]
  \mathbf{r_2} &=-{\mu\over m_2}\mathbf r
\end{align}

牛顿力学为开普勒定律提供了数学解释。
\subsection{开普勒第一定律}
{\bf 行星绕太阳运动的轨道是一个椭圆,太阳在椭圆的一个焦点上}。

\begin{equation}
  r={L^2/\mu^2 \over GM(1+e\cos\theta)}
  \label{eq:kepler1}
\end{equation}

式\ref{eq:kepler1}为极坐标下的圆锥曲线,{\bf 折合质量受引力作用绕质心运动的轨迹是圆锥曲线(抛物线、双曲线、椭圆)}。

\begin{equation}
  L=\mu\sqrt{GMa(1-e^2)}
\end{equation}

其中$L$在圆周运动($e=0$)时最大,抛物线运动($e\rightarrow 1$)时趋于0。

\subsection{开普勒第二定律}
{\bf 行星和太阳的连线在相同时间间隔内扫过的面积相等}。

\begin{equation}
  v^2=G(m_1+m_2)({2\over r}-{1\over a})
  \label{eq:kepler2}
\end{equation}


\subsection{开普勒第三定律}
{\bf (周期)平方(半长轴)反比律,或调和定律}

\begin{equation}
  P^2={4\pi^2 \over G(m_1+m_2)}a^3\;\text{国际单位制}
\end{equation}

方便起见,有时可写成:
\begin{equation}
  p^2={1\over (m_1+m_2)}a^3
  \label{eq:kepler3}
\end{equation}
其中$P$的单位是年,$a$的单位是AU,$m$的单位是太阳质量

\section{维里定律}
{\bf 对于引力束缚的平衡系统,平均动能总是平均势能的一半}
\begin{equation}
  -2\langle K \rangle=\langle U \rangle
  \label{virial}
\end{equation}

