\chapter{行星系统的形成}
前面讲到恒星是通过分子云塌缩形成的,但是如果分子云在一开始具有角动量(很普遍),那么在塌缩过程会形成一个吸积盘。

吸积盘会导致最终只有核心部分会形成恒星,而周围的物质会形成\textbf{残骸盘}。而在残骸盘冷却的过程中,一些尘埃颗粒会吸附周围的物质,形成质量越来越大的球,表面积也越大,越容易碰撞和吸附到新物质。

当球的质量大到可以用引力吸引周围物质时,便成为了\textbf{星子},星子体积与小行星大小相当。然后星子会进一步清扫轨道内和附近的物质,形成较大的天体,最终形成行星。

为了描述星子清扫周围物质的能力,我们可以假设如果微粒绕星子公转的周期等于星子绕太阳公转的周期,那么微粒就会被星子引力束缚,这时的半径被称为\textbf{希尔半径}:
\begin{equation}
  R_H=\left({M\over M_\odot}\right)^{1/3}a=\left({R\over R_\odot}\right)\left({\rho \over \rho_\odot}\right)^{-1/3}a={R\over \alpha}
\end{equation}

其中$a$是星子的轨道半长轴,$R$是微粒到星子的距离,计算时可认为是星子的大小。

